%% SECTION 1 %%
\section*{Introduction}

A majority of Americans experience stress in their daily lives \citep{apa2017stress}. Thus, an important goal of psychological research is to evaluate techniques that promote stress reduction and relaxation. Two techniques that have been associated with reduced stress and increased relaxation in psychotherapy contexts are guided imagery and progressive muscle relaxation \citep{mcguigan2007relaxation}. \emph{Guided imagery} aids individuals in connecting their internal and external experiences, allowing them, for example, to feel calmer externally because they practice thinking about calming imagery. \emph{Progressive muscle relaxation} involves diaphragmatic breathing and the tensing and releasing of 16 major muscle groups; together these be-

\begin{Figure}
    \centering
    \includegraphics[width=0.9\linewidth]{example-image}
    \captionof{figure}{
        \dummy{Image caption}.\\
        \indent\emph{Note}. From {\color{RedOrange} or} Adapted from \qq{\dummy{Article Title},} by \dummy{Initials}.~\dummy{Last Name}, \dummy{Year}, \emph{\dummy{Journal Name}, \dummy{Volume}}(\dummy{Issue}), p.~\dummy{Page Number} (\dummy{URL} {\color{RedOrange} or} \dummy{DOI}). \dummy{Copyright Statement}.
        }
    \label{fig: small-image}
\end{Figure}

\noindent haviors lead individuals to a more relaxed state \citep{jacobson1938relaxation, trakhtenberg2008imagery}. Guided imagery and progressive muscle relaxation are both cognitive behavioral techniques \citep{yalom2005theory} in which individuals focus on the relationship among thoughts, emotions, and behaviors \citep{white2000introduction}.

Group psychotherapy effectively promotes positive treatment outcomes in patients in a cost-effective way. Its efficacy is in part attributable to variables unique to the group experience of therapy as compared with individual psychotherapy \citep{bottomley1996group, yalom2005theory}. That is, the group format helps participants feel accepted and better understand their common struggles; at the same time, interactions with group members provide social support and models of positive behavior \citep{yalom2005theory}. Thus, it is useful to examine how stress reduction and relaxation can be enhanced in a group context.

The purpose of this literature review is to examine the research base on guided imagery and progressive muscle relaxation in group psychotherapy contexts. I provide overviews of both guided imagery and progressive muscle relaxation, including theoretical foundations and historical context. Then I examine guided imagery and progressive muscle relaxation as used on their own as well as in combination as part of group psychotherapy \citep[see][for more]{baider1994progressive}. Throughout the review, I highlight themes in the research. Finally, I end by pointing out limitations in the existing literature and exploring potential directions for future research.
