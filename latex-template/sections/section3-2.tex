%% SUBSECTION 3.2 %%
\subsection*{Progressive Muscle Relaxation in Group Psychotherapy.}

Limited, but compelling, research has examined progressive muscle relaxation within group psychotherapy. Progressive muscle relaxation has been used in outpatient and inpatient hospital settings to reduce stress and physical symptoms \citep{peterson2011relaxation}. For example, the U.S. Department of Veterans Affairs integrates progressive muscle relaxation into therapy skills groups \citep{hardy2017mindfulness}. The goal is for group members to practice progressive muscle relaxation throughout their inpatient stay and then continue the practice at home to promote ongoing relief of symptoms \citep{yalom2005theory}.

\citet{yu2004relaxation} examined the effects of multimodal progressive muscle relaxation on psychological distress in 121 elderly patients with heart failure. Participants were randomized into experimental and control groups. The experimental group received biweekly group sessions on progressive muscle relaxation, as well as tape-directed self-practice and a revision workshop. The control group received follow-up phone calls as a placebo. Results indicated that the experimental group exhibited significant improvement in reports of psychological distress compared with the control group. Although this study incorporated a multimodal form of progressive muscle relaxation, the experimental group met biweekly in a group format; thus, the results may be applicable to group psychotherapy.

Progressive muscle relaxation has also been examined as a stress-reduction intervention with large groups, albeit not therapy groups. \citet{rausch2006meditation} exposed a group of 387 college students to 20 min of either meditation, progressive muscle relaxation, or waiting as a control condition. Students exposed to meditation and progressive muscle relaxation recovered more quickly from subsequent stressors than did students in the control condition. \citet{rausch2006meditation} concluded the following: \qq{A mere 20 min of these group interventions was effective in reducing anxiety to normal levels (...) merely 10 min of the interventions allowed [the high-anxiety group] to recover from the stressor. Thus, brief interventions of meditation and progressive muscle relaxation may be effective for those with clinical levels of anxiety and for stress recovery when exposed to brief, transitory stressors.} Thus, even small amounts of progressive muscle relaxation can be beneficial for people experiencing anxiety.
