%% SUBSECTION 3.1 %%
\subsection*{Features of Progressive Muscle Relaxation.}

Progressive muscle relaxation involves diaphragmatic or deep breathing and the tensing and releasing of muscles in the body \citep{jacobson1938relaxation}. Edmund \citeauthor{jacobson1938relaxation} developed progressive muscle relaxation in 1929 \citep[as cited in][]{peterson2011relaxation} and directed participants to practice progressive muscle relaxation several times a week for a year. After examining progressive muscle relaxation as an intervention for stress or anxiety, Joseph Wolpe (1960; as cited in \cite{peterson2011relaxation}) theorized that relaxation was a promising treatment. In \citeyear{bernstein1973progressive}, \citeauthor{bernstein1973progressive} created a manual for helping professionals to teach their clients progressive muscle relaxation, thereby bringing progressive muscle relaxation into the fold of interventions used in cognitive behavior therapy. In its current state, progressive muscle relaxation is often paired with relaxation training and described within a relaxation framework \citep[see][for more]{freebird2012progressive}.

Research on the use of progressive muscle relaxation for stress reduction has demonstrated the efficacy of the method \citep{mcguigan2007relaxation}. As clients learn how to tense and release different muscle groups, the physical relaxation achieved then influences psychological processes \citep{mccallie2006progressive}. For example, progressive muscle relaxation can help alleviate tension headaches, insomnia, pain, and irritable bowel syndrome. This research demonstrates that relaxing the body can also help relax the mind and lead to physical benefits.
