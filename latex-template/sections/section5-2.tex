%% SUBSECTION 5.2 %%
\subsection*{Directions for Future Research.}

There are additional considerations when interpreting the results of previous studies and planning for future studies of these techniques. For example, a lack of control groups and small sample sizes have contributed to low statistical power and limited the generalizability of findings. Although the current data support the efficacy of psychotherapy groups that integrate guided imagery and progressive muscle relaxation, further research with control groups and larger samples would bolster confidence in the efficacy of these interventions. In order to recruit larger samples and to study participants over time, researchers will need to overcome challenges of participant selection and attrition. These factors are especially relevant within hospital settings because high patient turnover rates and changes in medical status may contribute to changes in treatment plans that affect group participation (L. Plum, personal communication, March 17, 2019). Despite these challenges, continued research examining guided imagery and progressive muscle relaxation interventions within group psychotherapy is warranted \citep{scherwitz2005therapy}. The results thus far are promising, and further investigation has the potential to make relaxation techniques that can improve people’s lives more effective and widely available.
